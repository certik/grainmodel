\chapter{FEM}

\section{Introduction}

We have a very simple model: we start with the continuity equation:
$$\nabla\cdot{\bf j}=-{\partial\rho\over\partial t}$$
and using the Ohm's law ${\bf j}=\sigma{\bf E}=-\sigma\nabla\varphi$ and setting
${\partial\rho\over\partial t}=0$ we get:
$$\nabla\cdot\sigma\nabla\varphi=0$$

More sophisticated model can be found at:

http://www.wias-berlin.de/project-areas/micro-el/power/index.html.en

\section{Weak Formulation of the Poisson Equation}

The Poisson equation is:
$$\nabla\cdot\lambda\nabla\varphi=-f$$
and the corresponding weak formulation is
$$\int\lambda\nabla\varphi\cdot\nabla v\,\d V= \int fv\,\d V+
\oint\lambda{\d\varphi\over\d n}v\,\d S\,,\no{w}$$
where ${\d\varphi\over\d n}\equiv\nabla\varphi\cdot{\bf n}$
The problem reads: find a function $\varphi$ such that \rno{w} holds for every
$v$.


\section{Finite Elements}

We choose a basis $\phi_i$ and substitute $\phi_i$ for $v$ and expand
$\psi=\sum q_j\phi_j$
$$\left(\int{\hbar^2\over2m}\nabla\phi_j\cdot\nabla\phi_i\,\d V+
\int\phi_iV\phi_j\,\d V\right)q_j
=
\left(\int E\phi_j\phi_i\,\d V\right)q_j
+\oint{\hbar^2\over2m}{\d\psi\over\d n}\phi_i\,\d S\,,\no{fem}$$
which can be written in a matrix form
$$\left(K_{ij}+V_{ij}\right)q_j=EM_{ij}q_j+F_i\,,$$
where
$$\eqalign{
V_{ij}&=\int\phi_iV\phi_j\,\d V\,,\cr
M_{ij}&=\int\phi_i\phi_j\,\d V\,,\cr
K_{ij}&={\hbar^2\over2m}\int\nabla\phi_i\cdot\nabla\phi_j\,\d V\,,\cr
F_i&={\hbar^2\over2m}\oint{\d\psi\over\d n}\phi_i\,\d S\,.\cr
}$$
Usually we set $F_i=0$.

We decompose the domain into elements and compute the integrals as the sum over
elements. For example:
$$K_{ij}=\sum_{E\in elements} K_{ij}^E$$
where $K_{ij}^E$ is the integral over one element only
$$
K_{ij}^{E}=\int{\hbar^2\over2m}\nabla\phi_j\cdot\nabla\phi_i\,\d V^{E}\approx
\sum_{q=0}^{N_q-1}{\hbar^2\over2m}\,\nabla\phi_i(x_q)\cdot\nabla\phi_j(x_q)\,
w_q|\det J(\hat x_q)|\,.
$$
The integral is computed numerically using a Gauss integration: $x_q$ are Gauss
points (there are $N_q$ of them), $w_q$ is the weight of each point, and the
Jacobian $|\det J(\hat x_q)|$ is there because we are actually computing the
integral on the reference element instead in the real space.

The surface integrals are computed similarly.
