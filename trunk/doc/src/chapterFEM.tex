\chapter{FEM}

\section{Introduction}

We have a very simple model: we start with the continuity equation:
$$\nabla\cdot{\bf j}=-{\partial\rho\over\partial t}$$
and using the Ohm's law ${\bf j}=\sigma{\bf E}=-\sigma\nabla\varphi$ and setting
${\partial\rho\over\partial t}=0$ we get:
$$\nabla\cdot\sigma\nabla\varphi=0$$

More sophisticated model can be found at:

http://www.wias-berlin.de/project-areas/micro-el/power/index.html.en

\section{Weak Formulation of the Poisson Equation}

The Poisson equation is:
$$\nabla\cdot\lambda\nabla\varphi=-f\no{poisson}$$
Usual procedure of solving such equations is to
multiply the equation with a test function $v$ and integrate over some
volume:
$$\int\nabla\cdot\lambda\nabla\varphi v \d V=-\int f v \d V\no{sim}$$
The problem reads: find a function $\varphi$ such that \rno{sim} holds for every
$v$. We then choose a basis $\phi_i$ (doesn't have to be normalized or
orthogonal) and expand $\varphi=\sum_j q_j \phi_j$ and set $v=\phi_i$:
$$\int\nabla\cdot\lambda\nabla\phi_j\phi_i \d V \, q_j=
-\int f \phi_i \d V\no{sim2}$$
which is of the form
$Aq=F$, but it has the disadvantage that the matrix $A$ is not symmetric
(solving is more difficult) and second that the basis $\phi_i$ needs to have
nonzero second derivatives (otherwise the matrix $A$ would vanish). Thus it is
convenient to rewrite \rno{sim} into this form (using the identity
$(\nabla\cdot\lambda\nabla u)v=-\lambda\nabla u \cdot \nabla v+
\nabla\cdot(\lambda v\nabla u)$
and Gauss theorem):
$$\int\lambda\nabla\varphi\cdot\nabla v\,\d V= \int fv\,\d V+
\oint\lambda{\d\varphi\over\d n}v\,\d S\,,\no{w}$$
where ${\d\varphi\over\d n}\equiv\nabla\varphi\cdot{\bf n}$. This is called
the weak formulation. In the basis $\phi_i$:
$$\int\lambda\nabla\phi_j\cdot\nabla \phi_i\,\d V\,q_j= \int f\phi_i\,\d V+
\oint\lambda{\d\varphi\over\d n}\phi_i\,\d S\,,\no{w2}$$
Mathematicians are used to saying that \rno{w} is more general than
\rno{poisson} or \rno{sim}, but this is of course not correct.

\section{Finite Elements}

We start with the \rno{w2},
which can be written in a matrix form
$$K_{ij}q_j=F_i\,,$$
where
$$\eqalign{
K_{ij}&=\int\lambda\nabla\phi_i\cdot\nabla\phi_j\,\d V\,,\cr
F_i&=\int f\phi_i\,\d V+\oint\lambda{\d\varphi\over\d n}\phi_i\,\d S\,.\cr
}$$
Usually we set $F_i=0$.

We decompose the domain into elements and compute the integrals as the sum over
elements. For example:
$$K_{ij}=\sum_{E\in elements} K_{ij}^E$$
where $K_{ij}^E$ is the integral over one element only
$$
K_{ij}^{E}=\int\lambda\nabla\phi_i\cdot\nabla\phi_j\,\d V^{E}\approx
\sum_{q=0}^{N_q-1}\lambda\nabla\phi_i(x_q)\cdot\nabla\phi_j(x_q)\,
w_q|\det J(\hat x_q)|\,.
$$
The integral is computed numerically using a Gauss integration: $x_q$ are Gauss
points (there are $N_q$ of them), $w_q$ is the weight of each point, and the
Jacobian $|\det J(\hat x_q)|$ is there because we are actually computing the
integral on the reference element instead in the real space.

The surface integral in $F_i$ is decomposed as a sum over boundary element
faces and computed using a 2D Gauss point integration.

\section{Boundary Conditions}

The Neumann boundary condition means to impose some condition on the normal
derivative of the solution on the boundary, which is achieved by setting
${\d\varphi\over\d n}$ in the surface integral
$\oint\lambda{\d\varphi\over\d n}\phi_i\,\d S$.

The Dirichlet boundary condition means to impose a condition $q_i=
u_0({\bf r}_i)$ where
$u_0({\bf r}_i)$ is the value of solution at the boundary at the position 
${\bf f}_i$ of
the $i$th node. This in theory could be achieved by setting
$K_{ij}=\delta_{ij}$ and $F_i=u_0$ (for all $i$ and $j$ in the $i$th row and
$i$th column), but it would be necessary to set the whole whole row and column
to $0$ (and $K_{ii}=1$).
So in practice we use an effectively equivalent procedure: we only set
$K_{ii}=P$, $F_i=u_0 P$, for some sufficiently large number $P$ (penalty).

Another effectively equaivalent procedure is to add the following equation 
to \rno{w2}:
$$P\int\phi_i\phi_j\d S q_j = P\int u_0({\bf r}) \phi_i\d S$$
This equation is used for all $i$ and $j$ of elements we want to impose the
Dirichlet BC. This way allows us to impose the Dirichlet BC on the whole
element face, with contrast to the above procedure, which only set the
condition to some nodes.

\section{Current}

$${\bf j}=-\sigma\nabla\varphi=-q_i\sigma\nabla\phi_i$$
