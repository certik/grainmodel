\chapter{Appendix}

\section{Polar and Spherical Coordinates}

Polar coordinates (radial, azimuth) $(r,\phi)$ are defined by
$$\eqalign{
x&=r\cos\phi\cr
y&=r\sin\phi\cr
}$$
Spherical coordinates (radial, zenith, azimuth) $(\rho,\theta,\phi)$:
$$\eqalign{
x&=\rho\cos\phi\sin\theta\cr
y&=\rho\sin\phi\sin\theta\cr
z&=\rho\cos\theta\cr
}$$

\section{Delta Function}

${\bf\hat r}$ is the unit vector in 3D and we can label it using
spherical coordinates ${\bf\hat r}={\bf\hat r}(\theta,\phi)$. We can also
express it in cartesian coordinates as 
${\bf\hat r}(\theta,\phi)=(\cos\phi\sin\theta,\sin\phi\sin\theta,\cos\theta)$.

$$f({\bf\hat r'})=\int\delta({\bf\hat r}-{\bf\hat r'})f({\bf\hat r})\,\d
{\bf\hat r}\no{deltar}$$
Expressing $f({\bf\hat r})=f(\theta,\phi)$ as a function of $\theta$ and $\phi$
we have
$$f(\theta',\phi')=\int\delta(\theta-\theta')\delta(\phi-\phi')
f(\theta,\phi)\,\d\theta\d\phi\no{deltaangles}$$
Expressing \rno{deltar} in spherical coordinates we get
$$f(\theta',\phi')=\int\delta({\bf\hat r}-{\bf\hat r'})
f(\theta,\phi)\sin\theta\,\d\theta\d\phi$$
and comparing to \rno{deltaangles} we finally get
$$\delta({\bf\hat r}-{\bf\hat r'})={1\over\sin\theta}
\delta(\theta-\theta')\delta(\phi-\phi')$$
In exactly the same manner we get
$$\delta({\bf r}-{\bf r'})=\delta({\bf\hat r}-{\bf\hat r'})
{\delta(\rho-\rho')\over\rho^2}$$

\section{Variations and functional derivatives}

Functional derivatives are a common source of confusion and especially the
notation. The reason is similar to the delta function --- it's due to rigid
mathematicians, who refuse to use a clear and exact notation and unfortunately
they are allowed to teach the basic calculus to physics students.

\def\h{{\rm h}}
\def\a{{\rm a}}
Let's have $\x=(x_1,x_2,\dots,x_N)$. The function $f(\x)$ assigns a number
to each $\x$. We define a differential of $f$ as
$$\d f\equiv \left.{\d\over\d\varepsilon}f(\x+\varepsilon\h)
\right|_{\varepsilon=0}
=\lim_{\varepsilon\to0}
{f(\x+\varepsilon\h)-f(\x)\over\varepsilon}=\a\cdot\h$$
The last equality follows from the fact, that
$\left.{\d\over\d\varepsilon}f(\x+\varepsilon\h) \right|_{\varepsilon=0}$ is a
linear function of $\h$.
We define ${\partial f\over\partial x_i}$ as
$$\a\equiv\left({\partial f\over\partial x_1},{\partial f\over\partial x_2},
\dots,{\partial f\over\partial x_N}\right)$$
This also gives a formula for computing ${\partial f\over\partial x_i}$: we
set $h_j=\delta_{ij}h_i$ and
$${\partial f\over\partial x_i}=a_i=\a\cdot\h=\left.{\d\over\d\varepsilon}
f(\x+\varepsilon(0,0,\dots,1,\dots,0))\right|_{\varepsilon=0}=$$
$$=\lim_{\varepsilon\to0}
{f(x_1,x_2,\dots,x_i+\varepsilon,\dots,x_N)-f(x_1,x_2,\dots,x_i,\dots,x_N)
\over\varepsilon}$$
But this is just the way the partial derivative is usually defined.
Every variable can be treated as a function (very simple one):
$$x_i=g(x_1,\dots,x_N)=\delta_{ij}x_j$$
and so we define
$$\d x_i\equiv\d g=\d(\delta_{ij}x_j)=h_i$$
and thus we write $h_i=\d x_i$ and $\h=\d\x$ and 
$$\d f={\d f\over\d x_i}\d x_i$$
So $\d\x$ have two meanings --- it's either $\h=\x-\x_0$ (a finite change in
the independent variable $\x$) or a differential,
depending on the context. Even mathematicians use this notation.

Functional $F[f]$ assigns a number to each function $f(x)$. The variation is
defined as 
$$\delta F[f]\equiv\left.{\d\over\d\varepsilon}F[f+\varepsilon h]
\right|_{\varepsilon=0}=\int a(x)h(x)\d x$$
We define ${\delta F\over\delta f(x)}$ as
$$a(x)\equiv{\delta F\over\delta f(x)}$$
This also gives a formula for computing ${\delta F\over\delta f(x)}$: we
set $h(y)=\delta(x-y)$ and
$${\delta F\over\delta f(x)}=a(x)=\int a(y)\delta(x-y)\d y=
\left.{\d\over\d\varepsilon}F[f(y)+\varepsilon\delta(x-y)]
\right|_{\varepsilon=0}=$$
$$=\lim_{\varepsilon\to0}
{F[f(y)+\varepsilon\delta(x-y)]-F[f(y)]\over\varepsilon}$$
Every function can be treated as a functional (although a very simple one):
$$f(x)=G[f]=\int f(y)\delta(x-y)\d y$$
and so we define
$$\delta f\equiv\delta G[f]=
\left.{\d\over\d\varepsilon}G[f(x)+\varepsilon h(x)]
\right|_{\varepsilon=0}=
\left.{\d\over\d\varepsilon}(f(x)+\varepsilon h(x))
\right|_{\varepsilon=0}=
h(x)$$
thus we write $h=\delta f$ and
$$\delta F[f]=\int {\delta F\over\delta f(x)}\delta f(x)\d x$$
so $\delta f$ have two meanings --- it's either 
$h(x)=\left.{\d\over\d\varepsilon}(f(x)+\varepsilon h(x))
\right|_{\varepsilon=0}$
(a finite change in the
function $f$) or a variation of a functional, depending on the context.
Mathematicians never write $\delta f$ in the meaning of $h(x)$, they always
write the latter, but it's ridiculous, because it is completely analogous to
$\d\x$.

The correspondence between the finite and infinite dimensional case can be
summarized as:
$$\eqalign{
f(x_i) &\quad\Longleftrightarrow\quad F[f]\cr
\d f=0 &\quad\Longleftrightarrow\quad \delta F=0\cr
{\partial f\over\partial x_i}=0 &\quad\Longleftrightarrow\quad 
{\delta F\over\delta f(x)}=0\cr
f &\quad\Longleftrightarrow\quad F\cr
x_i &\quad\Longleftrightarrow\quad f(x)\cr
x &\quad\Longleftrightarrow\quad f\cr
i &\quad\Longleftrightarrow\quad x\cr
}$$

This notation allows us a very convinient computation, as shown in the
following examples. First, when computing a variation of some integral, when
can interchange $\delta$ and $\int$:
$$F[f]=\int K(x) f(x) \d x$$
$$\delta F=\delta \int K(x) f(x) \d x 
= \left.{\d\over\d\varepsilon}\int K(x) (f+\varepsilon h)\d
x\right|_{\varepsilon=0}=
\left.\int{\d\over\d\varepsilon} (K(x) (f+\varepsilon h))\d
x\right|_{\varepsilon=0}=$$
$$=\int\delta(K(x) f(x))\d x $$
In the expression $\delta(K(x) f(x))$ we must understand from the context if we
are treating it as a functional of $f$ or $K$. In our case it's a functional of
$f$, so we have $\delta(K f)=K\delta f$.

A few more examples:
$${\delta\over\delta f(t)}\int\d t'f(t')g(t')=
\left.{\d\over\d\varepsilon}\int\d t'(f(t')+\varepsilon\delta(t-t'))g(t')
\right|_{\varepsilon=0}=g(t)$$
$${\delta f(t')\over\delta f(t)}=
\left.{\d\over\d\varepsilon}(f(t')+\varepsilon\delta(t-t'))
\right|_{\varepsilon=0}=\delta(t-t')$$
$${\delta f(t_1)f(t_2)\over\delta f(t)}=
\left.{\d\over\d\varepsilon}(f(t_1)+\varepsilon\delta(t-t_1))
(f(t_2)+\varepsilon\delta(t-t_2))
\right|_{\varepsilon=0}=\delta(t-t_1)f(t_2)+f(t_1)\delta(t-t_2)$$
$${\delta\over\delta f(t)}\half\int\d t_1\d t_2K(t_1,t_2)f(t_1)f(t_2)=
\half\int\d t_1\d t_2K(t_1,t_2){\delta f(t_1)f(t_2)\over\delta f(t)}=$$
$$=\half\left(\int\d t_1 K(t_1,t)f(t_1)+\int\d t_2 K(t,t_2)f(t_2)\right)
=\int\d t_2 K(t,t_2)f(t_2)$$
The last equality follows from $K(t_1,t_2)=K(t_2,t_1)$ (any antisymmetrical
part of a $K$ would not contribute to the symmetrical integration).

\section{Spherical Harmonics}

Are defined by
$$Y_{lm}(\theta,\phi)=\sqrt{{2l+1\over4\pi}{(l-m)!\over(l+m)!}}\,P_l^m(\cos\theta)\,e^{im\phi}$$
where $P_l^m$ are associated Legendre polynomials defined by
$$P_l^m(x)=(-1)^m (1-x^2)^{m/2}{\d^m\over\d x^m} P_l(x)$$
and $P_l$ are Legendre polynomials defined by the formula
$$P_l(x)={1\over2^l l!}{\d^l\over\d x^l}[(x^2-1)^l]$$
They are ortonormal:
$$\int Y_{lm}\,Y^*_{l'm'}\,\d\Omega = 
\int_0^{2\pi}\int_0^{\pi}
Y_{lm}(\theta,\phi)\,Y^*_{l'm'}(\theta,\phi)\sin\theta\,\d\theta\,\d\phi = 
\delta_{mm'}\delta_{ll'}\no{Yorto}$$
and complete (both in the $l$-subspace and the whole space):
$$\sum_{m=-l}^l|Y_{lm}(\theta,\phi)|^2={2l+1\over4\pi}\no{lcomplete}$$
$$\sum_{l=0}^\infty\sum_{m=-l}^lY_{lm}(\theta,\phi)Y_{lm}^*(\theta',\phi')
={1\over\sin\theta}\delta(\theta-\theta')\delta(\phi-\phi')=
\delta({\bf\hat r}-{\bf\hat r'})\no{Ycomplete}$$
The relation \rno{lcomplete} is a special case of an addition theorem for
spherical harmonics
$$\sum_{m=-l}^lY_{lm}(\theta,\phi)Y_{lm}^*(\theta',\phi')=
{4\pi\over 2l+1}P_l(\cos\gamma) \no{lsum}
$$
where $\gamma$ is the angle between the unit vectors given by $(\theta,\phi)$
and $(\theta',\phi')$:
$$\cos\gamma=\cos\theta\cos\theta'+\sin\theta\sin\theta'\cos(\phi-\phi')$$

\section{Dirac Notation}

$$f(x')=\int\delta(x'-x)f(x)\d x$$
we define
$$\eqalign{
f(x)&\equiv\braket{x|f}\cr
f(x')&\equiv\braket{x'|f}\cr
\delta(x'-x)&\equiv\braket{x'|x}\cr
}$$
so we have
$$\braket{x'|f}=\int\braket{x'|x}\braket{x|f}\d x$$
we also define
$$\int\ket{x}\bra{x}\d x\equiv\one$$
and thus we can interpret $\ket{f}$ as a vector, $\ket{x}$ as a basis and
$\braket{x|f}$ as the coefficients in the basis expansion:
$$\ket{f}=\one\ket{f}=\int\ket{x}\bra{x}\d x\ket{f}=
\int\ket{x}\braket{x|f}\d x$$

Now, let's look at the spherical harmonics:
$$Y_{lm}({\bf\hat r})\equiv\braket{{\bf\hat r}|lm}$$
on the unit sphere, we have
$$\int\ket{\bf\hat r}\bra{\bf\hat r}\d{\bf\hat r}=
\int\ket{\bf\hat r}\bra{\bf\hat r}\d\Omega=\one$$
$$\delta({\bf\hat r}-{\bf\hat r'})=\braket{{\bf\hat r}|{\bf\hat r'}}$$
thus
$$\int_0^{2\pi}\int_0^{\pi}
Y_{lm}(\theta,\phi)\,Y^*_{l'm'}(\theta,\phi)\sin\theta\,\d\theta\,\d\phi = 
\int\braket{l'm'|{\bf\hat r}}\braket{{\bf\hat r}|lm}\d\Omega=
\braket{l'm'|lm}
$$
and from \rno{Yorto} we get
$$\braket{l'm'|lm}=\delta_{mm'}\delta_{ll'}$$
now
$$\sum_{lm}Y_{lm}(\theta,\phi)Y_{lm}^*(\theta',\phi')=
\sum_{lm}\braket{{\bf\hat r}|lm}\braket{lm|{\bf\hat r'}}
$$
from \rno{Ycomplete} we get
$$\sum_{lm}\braket{{\bf\hat r}|lm}\braket{lm|{\bf\hat r'}}=
\braket{{\bf\hat r}|{\bf\hat r'}}$$
so we have
$$\sum_{lm}\ket{lm}\bra{lm}=\one$$
so $\ket{lm}$ forms an ortonormal basis. Any function defined on the sphere
$f({\bf\hat r})$ can be written using this basis:
$$f({\bf\hat r})
=\braket{{\bf\hat r}|f}
=\sum_{lm}\braket{{\bf\hat r}|lm}\braket{lm|f}
=\sum_{lm}Y_{lm}({\bf\hat r})f_{lm}
$$
where
$$
f_{lm}=\braket{lm|f}=\int\braket{lm|{\bf\hat r}}\braket{{\bf\hat r}|f}\d\Omega
=\int Y_{lm}^*({\bf\hat r}) f({\bf\hat r})\d\Omega
$$
If we have a function $f({\bf r})$ in 3D, we can write it as a function of
$\rho$ and ${\bf\hat r}$ and expand only with respect to the variable
${\bf\hat r}$: 
$$f({\bf r})=f(\rho{\bf\hat r})\equiv g(\rho,{\bf\hat r})=
\sum_{lm}Y_{lm}({\bf\hat r})g_{lm}(\rho)
$$
In Dirac notation we are doing the following: we decompose the space 
into the angular and radial part
$$\ket{{\bf r}}=\ket{{\bf\hat r}}\otimes\ket{\rho}
\equiv\ket{{\bf\hat r}}\ket{\rho}
$$
and write
$$f({\bf r})=\braket{{\bf r}|f}=\bra{{\bf\hat r}}\braket{\rho|f}=
\sum_{lm}Y_{lm}({\bf\hat r})\bra{lm}\braket{\rho|f}
$$
where
$$\bra{lm}\braket{\rho|f}=
\int\braket{lm|{\bf\hat r}}\bra{{\bf\hat r}}\braket{\rho|f}\d\Omega
=\int Y_{lm}^*({\bf\hat r}) f({\bf r})\d\Omega
$$
Let's calculate $\braket{\rho|\rho'}$
$$
\braket{{\bf r}|{\bf r'}}=\bra{\bf\hat r}\braket{\rho|\rho'}\ket{{\bf\hat r'}}
=\braket{{\bf\hat r}|{\bf\hat r'}}\braket{\rho|\rho'}
$$
so
$$\braket{\rho|\rho'}
={\braket{{\bf r}|{\bf r'}}\over\braket{{\bf\hat r}|{\bf\hat r'}}}
={\delta(\rho-\rho')\over\rho^2}$$
We must stress that $\ket{lm}$ only acts in the $\ket{{\bf\hat r}}$ space (not
the $\ket\rho$ space) which means that
$$\braket{{\bf r}|lm}
=\bra{\bf\hat r}\braket{\rho|lm}
=\braket{{\bf\hat r}|lm}\bra{\rho}
=Y_{lm}({\bf\hat r})\bra{\rho}
$$
and $V\ket{lm}$ leaves $V\ket\rho$ intact. Similarly,
$$\sum_{lm} \ket{lm}\bra{lm}=\one$$
is a unity in the $\ket{\bf\hat r}$ space only (i.e. on the unit sphere).

Let's rewrite the equation \rno{lsum}:
$$\sum_m\braket{{\bf\hat r}|lm}\braket{lm|{\bf\hat r'}}=
{4\pi\over 2l+1} P_l({\bf\hat r}\cdot{\bf\hat r'})$$
from which it follows the completeness relation for $P_l/(2l+1)$ (actually,
this is not a completeness relation\dots)
$$\sum_l{4\pi\over 2l+1} P_l({\bf\hat r}\cdot{\bf\hat r'})=
\sum_{lm}\braket{{\bf\hat r}|lm}\braket{lm|{\bf\hat r'}}=
\braket{{\bf\hat r}|{\bf\hat r'}}=\delta({\bf\hat r}-{\bf\hat r'})
$$
this is the completeness relation
$$\sum_{l=0}^\infty {2l+1\over2}P_l(x')P_l(x)=\delta(x-x')$$
$$\sum_l {2l+1\over2}\braket{x'|P_l}\braket{P_l|x}=\braket{x'|x}$$
$$\sum_l \ket{P_l}{2l+1\over2}\bra{P_l}=\one$$
and
$$f(x)=\braket{x|f}=\sum_l\braket{x|P_l}{2l+1\over2}\braket{P_l|f}=
\sum_{l=0}^\infty f_l P_l(x)$$
where
$$f_l={2l+1\over2}\int_{-1}^1 P_l(x)f(x)\d x$$
which means that every function $f({\bf\hat r}\cdot{\bf\hat r'})$ 
can be expanded in terms of $P_l$:
$$f({\bf\hat r}\cdot{\bf\hat r'})=\sum_{l=0}^\infty f_l 
{4\pi\over 2l+1} P_l({\bf\hat r}\cdot{\bf\hat r'})$$
and using \rno{lsum}
$$f({\bf\hat r}\cdot{\bf\hat r'})=
\sum_{l=0}^\infty\sum_{m=-l}^l Y_{lm}({\bf\hat r}) f_l Y_{lm}^*({\bf\hat r'})
\no{fylm}
$$
