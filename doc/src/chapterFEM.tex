\chapter{FEM}

\section{Introduction}

This chapter explains FEM and gives concrete formulas which are needed in the
calculation.

\section{Weak Formulation of the Schr\"odinger Equation}

One particle Schr\"odinger equation is
$$\left(-{\hbar^2\over2m}\nabla^2 + V\right)\psi=E\psi\,.$$
We multiply both sides by a test function $v$
$$-\left({\hbar^2\over2m}\nabla^2\psi\right)v=(E-V)\psi v\,,$$
and integrate over the whole volume we are interested in
$$\int-\left({\hbar^2\over2m}\nabla^2\psi\right)v\,\d V=\int(E-V)\psi v\,\d
V\,,\no{1}$$
and using the vector identity
$$-\left(\nabla^2\psi)\right)v=\nabla \psi\cdot
\nabla v - \nabla\cdot\left((\nabla \psi)v\right),$$
we rewrite the left hand side of \rno{1}
$$\int{\hbar^2\over2m}\nabla\psi\cdot\nabla v\,\d V=\int(E-V)\psi v\,\d
V+\int{\hbar^2\over2m}\nabla\cdot\left((\nabla \psi)v\right)\,\d V\,,$$
now we apply Gauss Theorem
$$\int{\hbar^2\over2m}\nabla\psi\cdot\nabla v\,\d V=\int(E-V)\psi v\,\d
V+\oint{\hbar^2\over2m}(\nabla \psi)v\cdot{\bf n}\,\d S\,,$$
and rewriting $\nabla\psi\cdot{\bf n}\equiv{\d\psi\over\d n}$
%$$\int{\hbar^2\over2m}\nabla\psi\cdot\nabla v\,\d V=\int(E-V)\psi v\,\d
%V+\oint{\hbar^2\over2m}{\d\psi\over\d n}v\,\d S\,,\no{w}$$
$$\int{\hbar^2\over2m}\nabla\psi\cdot\nabla v\,\d V+ \int vV\psi\,\d V
=
\int E\psi v\,\d V + \oint{\hbar^2\over2m}{\d\psi\over\d n}v\,\d S\,,\no{w}$$
which is the weak formulation. The problem reads: find a function $\psi$ such
that \rno{w} holds for every $v$.

\section{Finite Elements}

We choose a basis $\phi_i$ and substitute $\phi_i$ for $v$ and expand
$\psi=\sum q_j\phi_j$
$$\left(\int{\hbar^2\over2m}\nabla\phi_j\cdot\nabla\phi_i\,\d V+
\int\phi_iV\phi_j\,\d V\right)q_j
=
\left(\int E\phi_j\phi_i\,\d V\right)q_j
+\oint{\hbar^2\over2m}{\d\psi\over\d n}\phi_i\,\d S\,,\no{fem}$$
which can be written in a matrix form
$$\left(K_{ij}+V_{ij}\right)q_j=EM_{ij}q_j+F_i\,,$$
where
$$\eqalign{
V_{ij}&=\int\phi_iV\phi_j\,\d V\,,\cr
M_{ij}&=\int\phi_i\phi_j\,\d V\,,\cr
K_{ij}&={\hbar^2\over2m}\int\nabla\phi_i\cdot\nabla\phi_j\,\d V\,,\cr
F_i&={\hbar^2\over2m}\oint{\d\psi\over\d n}\phi_i\,\d S\,.\cr
}$$
Usually we set $F_i=0$.

We decompose the domain into elements and compute the integrals as the sum over
elements. For example:
$$K_{ij}=\sum_{E\in elements} K_{ij}^E$$
where $K_{ij}^E$ is the integral over one element only
$$
K_{ij}^{E}=\int{\hbar^2\over2m}\nabla\phi_j\cdot\nabla\phi_i\,\d V^{E}\approx
\sum_{q=0}^{N_q-1}{\hbar^2\over2m}\,\nabla\phi_i(x_q)\cdot\nabla\phi_j(x_q)\,
w_q|\det J(\hat x_q)|\,.
$$
The integral is computed numerically using a Gauss integration: $x_q$ are Gauss
points (there are $N_q$ of them), $w_q$ is the weight of each point, and the
Jacobian $|\det J(\hat x_q)|$ is there because we are actually computing the
integral on the reference element instead in the real space.

The surface integrals are computed similarly.
